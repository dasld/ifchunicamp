% -- capa
\newcommand*\imprimircapa{%
  \begingroup
  \microtypesetup{activate=false}%
  \currentpdfbookmark{Capa}{capa}% <- antes do \selectlanguage!
  \selectlanguage{brazilian}%
  \parindent=\z@
  \sffamily
  \centering
  %\offinterlineskip
  \nointerlineskip

% os campos realmente indispensáveis:
\@for\next:={autor,titulo,data,preambulo,orientador/cargo,orientador/nome}\do{%
  \ifch@check@campo{\next}%
}%

% CAPA: LOGOTIPO DA UNICAMP
\begingroup
  \setbox\@tempboxa=\vbox{\halign{%
    \bfseries\hfill##\hfill\cr  % duplo # por estarmos dentro de uma definição de macro
    UNIVERSIDADE ESTADUAL DE CAMPINAS\cr
    INSTITUTO DE FILOSOFIA E CIÊNCIAS HUMANAS\cr
  }}%
  % a altura mínima do logotipo é 10mm
  % a altura máxima "em um formulário" é 6% do maior lado do papel
  % 0,06*297mm = 17,82mm
  % https://www.unicamp.br/logotipo
  \ifdim\ifch@altura@logo<\z@ \global\ifch@altura@logo=\dimexpr \ht\@tempboxa + \dp\@tempboxa \relax\fi
  \ifdim\ifch@altura@logo<10mm\global\ifch@altura@logo=10mm\fi
  \ifdim\ifch@altura@logo>17.82mm\global\ifch@altura@logo=17.82mm\fi
  \ifch@put@logo  % isso é \relax quando vigora a opção "sem logo"
  \leavevmode\box\@tempboxa\par
\endgroup

% CAPA: AUTOR
\vskip 3\baselineskip
\MakeUppercase{\large\ifchcampo{autor}}\par

% CAPA: TÍTULO
\vfill
\parbox[t]\textwidth{%
  \centering\LARGE
  \MakeUppercase{\bfseries\ifchcampo{titulo}}\ifch@subtitle@line
  %
  \ifch@ifnotempty{\ifchcampo{titulo secundario}}{%
    \vskip\baselineskip
    \MakeUppercase{\bfseries\ifchcampo{titulo secundario}}\ifch@subtitle@line*%
  }%
}\par

% CAPA: LOCAL E DATA
\vfill
\large
CAMPINAS\par
\ifchcampo{data}\par
\endgroup
  \clearpage
}



% -- folha de rosto
\newcommand*\imprimirfolhaderosto{%
  \begingroup
  \microtypesetup{activate=false}%
  \currentpdfbookmark{Folha de rosto}{folhaderosto}% <- antes do \selectlanguage!
  \selectlanguage{brazilian}%
  \parindent=\z@
  \centering
  \nointerlineskip

% FOLHA DE ROSTO: AUTOR
\MakeUppercase{\large\ifchcampo{autor}}\par

\vfill
\parbox[t]\textwidth{%
  \centering
  % FOLHA DE ROSTO: TÍTULO
  \begingroup
  \sffamily\Large%\doublespacing
  \MakeUppercase{\ifchcampo{titulo}}\ifch@subtitle@line
  \ifch@ifnotempty{\ifchcampo{titulo secundario}}{%
    \vskip\baselineskip
    \MakeUppercase{\ifchcampo{titulo secundario}}\ifch@subtitle@line*%
  }%
  \endgroup

% FOLHA DE ROSTO: PREÂMBULO
\ifch@ifnotempty{\ifchcampo{preambulo}}{%
  \vskip .12\textheight
  \hspace*{\fill}%
  \parbox{.5\textwidth}{\singlespacing\expanded{\ifchcampo{preambulo}}}\par
}

% FOLHA DE ROSTO: PREÂMBULO ESTRANGEIRO
\ifch@ifnotempty{\ifchcampo{preambulo secundario}}{%
  \vskip \baselineskip
  \hspace*{\fill}%
  \parbox{.5\textwidth}{\singlespacing\expanded{\ifchcampo{preambulo secundario}}}\par
}

% flushleft é um environment; \raggedright é um switch
\raggedright
% FOLHA DE ROSTO: ORIENTAÇÃO
\vskip 2\baselineskip
\textit{\ifchcampo{orientador/cargo}}:\enspace\ifchcampo{orientador/nome}%
\ifch@ifnotempty{\ifchcampo{coorientador/nome}}{%
  \par\textit{\ifchcampo{coorientador/cargo}}:\enspace\ifchcampo{coorientador/nome}%
}\par

% FOLHA DE ROSTO: VERSÃO FINAL
\vskip 2\baselineskip
\@tempswatrue  % true se o texto de versão final está vazio
\ifch@ifnotempty{\ifchcampo{texto de versao final}}{\@tempswafalse}%
\ifx\ifch@put@vfinal\@firstofone  % precisamos imprimir a frase de versão final...
    \if@tempswa                   % ...mas o texto de versão final está vazio!
      \ClassError{ifchunicamp}{%
        Não há texto de versão final. Por favor,
        informe o texto de versão final com o comando
        \protect\SetKeys[ifchunicamp]{texto de versao final = {...}}%
      }{}% sem ajuda extra
    \fi
\fi
% imprima (normalmente ou com \phantom):
\ifch@put@vfinal{%
  \parbox{.5\textwidth}{%
    \footnotesize
    \singlespacing
    \MakeUppercase{\expanded{\ifchcampo{texto de versao final}}}%
  }%
}%
}% fim do grande \parbox
\par

% FOLHA DE ROSTO: LOCAL E DATA
\vfill
\large
CAMPINAS\par  \ifchcampo{data}\par
\endgroup
  \clearpage
}
