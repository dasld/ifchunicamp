% -- declaração das chaves
\newcommand*\ifch@bookoptions{draft,final,leqno,fleqn}


\ExplSyntaxOn

\prop_new:N \ifch@keys

\newcommand*\ifchcampo[1]{
  \prop_item:Ne \ifch@keys {#1}
}

\newcommand*\ifch@check@campo[1]{
  \begingroup
  \sbox0{\ifchcampo{#1}}
  \ifdim0pt<\wd0\relax
    \else \ClassWarning{ifchunicamp}{O~campo~"#1"~não~foi~informado}
  \fi
  \endgroup
}

\newcommand*\ifch@slashed@kv{
  % \l_keys_key_str é a parte após o último "/" no nome
  % \l_keys_path_str é todo o path, com a parte do ifchunicamp no começo
  \begingroup
  % \@nil é apenas um delimitador, usado em @car e @cdr
  \def\ifch@temp##1/##2\@nil{##2}
  \prop_gput:Nee \ifch@keys { \expandafter\ifch@temp\l_keys_path_str\@nil } { \l_keys_value_tl }
  \endgroup
}


\keys_define:nn { ifchunicamp }{
  % 1) opção "logo"
  logo .code:n = {
    \ifch@if@image@exists{#1}
      { \gdef\ifch@logo@file{#1} }
      { \ifch@bad@logo{#1} }
  },
  logo .value_required:n = true,
  % 2) opção "altura do logo"
  altura~do~logo .dim_gset:N = \ifch@altura@logo,
  altura~do~logo .value_required:n = true,
  % 3) opção "sem logo": omite o logo na capa
  sem~logo .code:n = { \global\let\ifch@put@logo=\relax },
  sem~logo .value_forbidden:n = true,
  % 4) opção "nao final": omite o bloco "ESTE TRABALHO CORRESPONDE À VERSÃO FINAL" na folha de rosto
  % usar \@gobble faria o conteúdo ser totalmente omitido,
  % mas queremos que ele ocupe espaço sem aparecer
  nao~final .code:n = { \global\let\ifch@put@vfinal=\phantom },
  nao~final .value_forbidden:n = true,
  % 5) opção "autor"
  autor .prop_gput:N = \ifch@keys,
  autor .value_required:n = true,
  % 6) opção "titulo"
  titulo .prop_gput:N = \ifch@keys,
  titulo .value_required:n = true,
  % 7) opção "subtitulo"
  subtitulo .prop_gput:N = \ifch@keys,
  subtitulo .value_required:n = true,
  % 8) opção "titulo secundario"
  titulo~secundario .prop_gput:N = \ifch@keys,
  titulo~secundario .value_required:n = true,
  % 9) opção "subtitulo secundario"
  subtitulo~secundario .prop_gput:N = \ifch@keys,
  subtitulo~secundario .value_required:n = true,
  % 10) opção "data"
  data .prop_gput:N = \ifch@keys,
  data .value_required:n = true,
  % 11) opção "preambulo"
  preambulo .prop_gput:N = \ifch@keys,
  preambulo .value_required:n = true,
  % 12) opção "preambulo secundario"
  preambulo~secundario .prop_gput:N = \ifch@keys,
  preambulo~secundario .value_required:n = true,
  % 13) opção "texto de versao final"
  texto~de~versao~final .prop_gput:N = \ifch@keys,
  texto~de~versao~final .value_required:n = true,
  % 14) opção "orientador"
  orientador / nome .code:n = \ifch@slashed@kv,
  orientador / nome .value_required:n = true,
  orientador / cargo .code:n = \ifch@slashed@kv,
  orientador / cargo .value_required:n = true,
  % 15) opção "coorientador"
  coorientador / nome .code:n = \ifch@slashed@kv,
  coorientador / nome .value_required:n = true,
  coorientador / cargo .code:n = \ifch@slashed@kv,
  coorientador / cargo .value_required:n = true,
  % -> chave desconhecida
  unknown .code:n = {
    \@expandtwoargs\in@{\CurrentOption}{\ifch@bookoptions}
    \ifin@
      \PassOptionsToClass{\CurrentOption}{book}
    \else
      \ClassError{ifchunicamp}{A~opção~"\CurrentOption"~não~é~permitida}{}
    \fi
  },
}  % fim de \keys_define:nn


\ExplSyntaxOff
\ProcessKeyOptions[ifchunicamp]\relax

